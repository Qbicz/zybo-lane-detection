\section{Model programowy transformaty Hougha}

\blindtext

\subsection{Model funkcjonalny z użyciem OpenCV}

\texttt{hough()}

screeny z programu

% 3 obrazki obok siebie
\begin{figure}[!htb]
\minipage{0.32\textwidth}
  \includegraphics[width=\linewidth]{img/road4.jpg}
  \caption{Obraz z kamery}\label{fig:awesome_image1}
\endminipage\hfill
\minipage{0.32\textwidth}
  \includegraphics[width=\linewidth]{img/canny_screen.png}
  \caption{Krawędzie znalezione metodą Canny}\label{fig:awesome_image2}
\endminipage\hfill
\minipage{0.32\textwidth}%
  \includegraphics[width=\linewidth]{img/hough_screen.png}
  \caption{Linie wybrane przez transformatę Hougha z głosowaniem}\label{fig:awesome_image3}
\endminipage
\end{figure}

\blindtext

%
%\begin{figure}[!htb]
%\centering
%\includegraphics[scale=1]{img/start2.png}
%\caption{Rozwiązanie przed optymalizacją z wektorem przełączeń $\tau = [0.015, 0.035, 0.055]$}
%\label{rys:start1}
%\end{figure}

\newpage
\subsection{Model stałoprzecinkowy w języku C++}

Klasa \texttt{fp.h}

\begin{figure}[!htb]
\centering
\includegraphics[scale=0.75]{img/fixed.png}
\caption{Wynik działania funkcji \texttt{Hough()} dla liczb stałoprzecinkowych}
\label{rys:fp}
\end{figure}

