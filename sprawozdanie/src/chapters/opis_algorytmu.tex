\section{Opis algorytmu}



\subsection{Wykrywanie krawędzi metodą Canny}

Zacytować pracę magisterską, nie opisujemy chyba?

\subsection{Rozpoznawanie linii - transformata Hougha}

Hough

%\begin{figure}[!htb]
%  \begin{center}
%    \includegraphics[scale=0.85]{img/maglev.PNG}
%    \caption{Stanowisko laboratoryjne magnetycznej lewitacji}
%  \end{center}
%  
%  \label{rys:maglev}
%\end{figure}

%\begin{equation} \label{maglev_rown}
%  \begin{cases}
%    \dot{x}_1 & = x_2 \\
%    \dot{x}_2 & = -e^{-x_1} \cdot x_3^2 + 1 \\
%    \dot{x}_3 & = -cx_3 + u
%    \end{cases}
%\end{equation}
%

%Współczynniki przeskalowania zebrano w tabeli.
%
%\begin{table}[!htb]
%  \centering
%  \begin{tabular}{|c|l|l|}
%  \hline
%  Współczynnik & Wartość \\
%  \hline
%  $\alpha$ & $0,00773746 m$ \\
%  \hline
%  $\beta$ & $0,275507681 m/s$ \\
%  \hline
%  $\gamma$ & $0,28890446065998 A$ \\
%  \hline
%  $\xi$ & $0,2808437120924$ \\
%  \hline
%  $\eta$ & $10,28701901522286 A/s$ \\
%  \hline
%  \end{tabular}
%  \caption{Parametry przeskalowania modelu}
%  \label{tab:idf}
%\end{table}


