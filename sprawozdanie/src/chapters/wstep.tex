\section{Wstęp}

Jazda samochodem jest niebezpieczna. W Polsce w 2015 roku miało miejsce 32 967 wypadków drogowych, w których zginęło 2 938 osób \cite{policja}. 82,8\% wypadków było spowodowanych przez kierowców.

Jednym z systemów, który może poprawić bezpieczeństwo na drogach jest ostrzeżenie o zmianie pasa ruchu. Sygnał dźwiękowy czy nawet aktywna zmiana toru jazdy samochodu może powstrzymać zasypiającego lub nieuważnego kierowcę przed zjechaniem do rowu, potrąceniem pieszego na poboczu czy spowodowaniem kolizji z innymi samochodami.

W celu wykrycia pasów ruchu i położenia samochodu względem jezdni wykorzystaliśmy wykrywanie krawędzi metodą Canny i następnie transformatę Hougha na otrzymanym obrazie binarnym w celu znalezienia linii malowań zbliżonych do prostych. W dalszej części opisano model programowy w OpenCV, model stałoprzecinkowy C++ oraz opis sprzętu na platformę Xilinx Zynq w języku Verilog.

