\section{Wykrywanie pasów ruchu - przegląd artykułów naukowych}

Projekt miał charakter badawczo-implementacyjny. Do uzyskania obrazu binarnego krawędzi wykorzystaliśmy algorytm Canny opracowany w pracy inżynierskiej \cite{inz-canny}. Następnie należało zdecydować w jaki sposób rozpoznawać linie i jak wybrany algorytm zrealizować w torze wizyjnym na platformie Xilinx Zynq-7000.

W pracy \cite{hardware-accelerator} opisano

%\begin{figure}[!htb]
%  \begin{center}
%    \includegraphics[width=14.5cm,trim=1.6cm 6.9cm 1.7cm 8.5cm,clip]
%    {img/exp_omega.pdf}
%  \end{center}
%  \caption{Eksperyment wyznaczenia charakterystyk prędkości śmigieł od napięcia na silnikach przy zablokowanych osiach}
%  \label{plot:exp1}
%\end{figure}





